% ************************** Thesis Abstract *****************************
% Use `abstract' as an option in the document class to print only the titlepage and the abstract.
\begin{abstract}
    
    There has been ubiquitous observations of wave-like motions in the solar atmosphere for decades and the presence of magneto-acoustic waves in magnetic structures in the solar atmosphere is well-documented.
    In this thesis, we aim to detect and identify magnetohydrodynamic (MHD) sausage waves in the lower solar atmosphere. 
    In order to achieve this, high-resolution ground-based data is taken from numerous solar telescopes.
    For this thesis, two sunspots and three magnetic pore were chosen as examples of MHD waveguides in the lower solar atmosphere.
    Combining the Wavelet Transform and Empirical Mode Decomposition, we determined characteristic periods within the cross-sectional area and intensity time series and several oscillations have been detected within these waveguides.
    Then, by applying the theory of linear MHD, we identified the mode of these oscillations and concluded that they can be classified as slow sausage MHD waves. 
    Further, we determined several key properties of these oscillations such as the radial velocity perturbation, magnetic field perturbation and vertical wavenumber using magneto-seismology.
    The estimated range of the related wavenumbers reveals that these oscillations are trapped within these magnetic structures.
    Our results suggest that the detected oscillations are standing harmonics, and, this allows us to estimate the expansion factor of the waveguides by employing magneto-seismology.
    Finally, we analysed Running Penumbral Waves (RPWs).
    RPWs have always thought to be radial wave propagation that occur within sunspots.
    Here, RPWs within a magnetic pore are observed for the first time and are interpreted as Upwardly Propagating Waves (UPWs) due to the lack of a penumbra that is required to support RPWs.
    These UPWs are also observed co-spatially and co-temporally within two elemental lines that sample the Transition Region and low corona.
    The observed UPWs are travelling at a horizontal velocity of around 17 $\pm$ 0.5 km s$^{-1}$ and a minimum vertical velocity of 42 $\pm$ 21 km s$^{-1}$.
    The estimated energy of the waves is around 150 W m$^{-2}$, which is on the lower bounds required to heat the quiet-Sun corona.
    
\end{abstract}
