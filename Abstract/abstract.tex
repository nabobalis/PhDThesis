% ************************** Thesis Abstract *****************************
\begin{abstract}
    There has been ubiquitous observations of wave-like motions in the solar atmosphere for decades and the presence of magneto-acoustic waves in magnetic structures in the solar atmosphere is well-documented.
    In this thesis, we aim to detect and identify magnetohydrodynamic (MHD) sausage waves in the lower solar atmosphere. 
    In order to achieve this, high-resolution data is taken from numerous solar telescopes.
    Overall, two sunspots and four magnetic pores were chosen as good examples of MHD waveguides in the lower solar atmosphere.
    The aim was to identify MHD wave modes in the cross-sectional area of these magnetic structures.
    To do this, the cross-sectional area and total intensity was measured through time, then this signal was analyzed with two signal analysis methods, namely, wavelets and Empirical Mode Decomposition.
    We determined several characteristic periods within the cross-sectional area and total intensity time series.
    To identify what MHD wave mode these oscillations are, previously derived linear MHD theory details that each MHD wave mode perturbs the cross-sectional area and total intensity differently.
    This phase difference is used to separate the possible MHD wave modes.
    These oscillations were identified as slow sausage MHD waves, as the phase difference between the cross-sectional area and total intensity was in-phase which is the signature of the slow sausage MHD waves.
    Further, we determined several properties of these oscillations such as the radial velocity perturbation, magnetic field perturbation and vertical wavenumber using magneto-seismology.
    The estimated range of the wavenumbers reveals that these oscillations are trapped within these magnetic structures and there is a very strong suggestion that these oscillations are standing harmonics.
    Which allowed the calculation of the expansion factor of the waveguides by employing further magneto-seismology.
    The final part was the analysis of Running Penumbral Waves (RPWs).
    Here, RPWs within a magnetic pore are observed for the first time and are interpreted as Upwardly Propagating Waves (UPWs) due to the lack of a penumbra that is required to support RPWs.
    These UPWs are also observed co-spatially and co-temporally within two elemental lines that sample the Transition Region and low corona.
    The estimated energy of the waves is around 150 W m$^{-2}$, which is on the lower bounds required to heat the quiet-Sun corona.
\end{abstract}
