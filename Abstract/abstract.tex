\begin{abstract}
    There have been ubiquitous observations of wave-like motions in the solar atmosphere for decades and the presence of magnetoacoustic waves in magnetic structures in the solar atmosphere is well-documented.
	By using high-resolution data sets taken from several solar telescopes, the aim was to identify magnetohydrodynamics (MHD) wave modes in the cross-sectional area of these magnetic structures.
	Two sunspots and four magnetic pores were chosen as good examples of MHD wave guides in the lower solar atmosphere.
    To achieve this aim, the cross-sectional area and total intensity was measured through time, then this signal was analysed with three signal analysis methods, namely, wavelets, empirical mode decomposition (EMD) and the fast Fourier transform (FFT).
    Many characteristic periods were found within the cross-sectional area and total intensity time series.
    To identify what MHD wave mode these oscillations are, previously derived linear MHD theory details that each MHD wave mode perturbs the cross-sectional area and total intensity differently.
    This phase difference is used to separate the possible MHD wave modes.
    These oscillations were identified as slow sausage MHD waves, as the phase difference between the cross-sectional area and total intensity was in phase which is the signature of slow sausage MHD waves.
    Furthermore, several properties of these oscillations such as the radial velocity perturbation, magnetic field perturbation and vertical wavenumber were determined using magneto-seismology.
    The calculated range of the wavenumbers reveals that these oscillations are trapped within these magnetic structures and are standing harmonics.
    This allowed the calculation of the expansion factor of the wave guides by employing further magneto-seismology theory.
    Finally was the analysis of Running Penumbral Waves (RPWs).
    Here, RPWs within a magnetic pore are observed for the first time and are interpreted as Upwardly Propagating Waves (UPWs) due to the lack of a penumbra that is required to support RPWs.
    These UPWs are also observed co-spatially and co-temporally within two emission lines that sample the Transition Region and low corona.
    The estimated energy of the waves is around 150 W m$^{-2}$, which is on the lower bounds required to heat the quiet Sun corona.
\end{abstract}