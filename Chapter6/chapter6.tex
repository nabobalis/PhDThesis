\graphicspath{{Chapter6/Figs/}}

\chapter{Conclusion}
\label{chapter6}
    
    \vspace*{\fill}\par
    \pagebreak

\section{Overview of the thesis}
    
	In this thesis, the results of two-dimensional image analysis of sunspots and magnetic pores in the lower solar atmosphere is detailed.
    These results are directly compared to theoretically derived results and indicate the ubiquitous presence of slow MHD sausage waves in the larger magnetic structures that inhabit the solar surface.
    
    In Chapter 2, the telescopes and their associated instruments that supplied the science ready data sets are described, along with brief data reduction details.
    Then the signal analysis methods used to determine the periods and phase difference were detailed.
    These being the Fast Fourier Transform (FFT), Wavelets and Empirical Mode Decomposition (EMD).
    Finally, the method used to measure the cross-sectional area and total intensity of the magnetic structures is examined.
    The reason for this is to understand what the effects of varying the sigma multiplier has on the results of the signal analysis, which is important before this method is used to analyse data sets.
    
    In Chapter 3, the cross-sectional area and total intensity is measured on two sunspots and one magnetic pore.
    By comparing the phase difference between the cross-sectional area and total intensity signals, which are calculated using the method described in Chapter 2.
    It was possible to find the ubiquitous presence of slow MHD sausage waves within these magnetic structures.
    This conclusion is reached because the measured phase difference was very close to 0$^\circ$, i.e., they were in phase.
    This is the signature of the slow MHD sausage wave within cylindrical magnetic flux tubes.
    
    In Chapter 4, the cross-sectional area and total intensity is calculated for two new magnetic pore data sets.
    The phase difference indicated that the oscillations are slow MHD sausage waves.
    The usage of magneto-seismology equations allowed the calculation of several properties of the detected oscillations within the two magnetic pores.
    These were, for example, the radial distance perturbation, radial velocity perturbation and magnetic field perturbations.
    The properties of these oscillations gave the impression of standing harmonics.
    However, it was possible to use magneto-seismology to demonstrate that these oscillations are in fact standing harmonics.
    This was accomplished by working out if density stratification or radial expansion of the flux can cause the observed period ratios.
    The calculated radial expansion was in good agreement with previous simulations and observations. 
    
    In Chapter 5, the focus shifted from analysing the cross-sectional of magnetic structures to the analysis of Running Penumbral Waves (RPWs) in a magnetic pore.
    RPWs have only observed in sunspot penumbras; however, RPW-like events were observed, for the first time, to occur around a magnetic pore.
    This observation is the final step in confirming that RPWs are a visual effect of MHD sausage waves following the curvature of magnetic field lines of sunspots and magnetic pores.
    These have been previously termed Upwardly Propagating Waves (UPWs).
    This visual effect is caused because the magnetic field becomes more radially inclined further from the umbral centre.
    These radially inclined fields have longer arc lengths and thus a wave following these field lines have to cover a larger distance before it can appear in the chromosphere.
    This creates a delay of the appearance of these waves at higher levels, causing this patten of radially outward propagating waves.
    These events are able to deliver a small amount of energy into the local corona, but not enough to heat the active corona.
       
\section{Summary of results}

	\subsection{Chapter 2}

    Chapter 2 started by detailing the various solar telescopes, space- and ground-based, utilised within this thesis.
    It covered the instruments that was either on board the space-based telescopes or attached to an optics bench for ground-based telescopes.
    From here, the three signal analysis methods employed are described: FFT, Wavelets and EMD.
    It detailed how each method works, strengths and weakness and why three methods were used instead of just one.
    
    The chapter finishes on a study of the principle method that is used to measure the cross-sectional area of the sunspots and magnetic pores analysed within this thesis.
    The motivation was to understand if the method output was heavily affected or dependent on the sigma multiplier value chosen.
    If this was the case, then this had to be known before this method could be used for any scientific analysis.    
    The sigma multiplier is a crucial factor as it will influence the end result, which is the cross-sectional area and total intensity signals.
    In order to undertake this study, high-quality ground-based data was used from the Dunn Solar Telescope (DST).
    The instruments used from the DST are the \textit{Interferometric Bidimensional Spectrometer} (IBIS) and the \textit{Rapid Oscillations in the Solar Atmosphere} (ROSA) instrument.
    There are two datasets, one for each instrument, with the telescope pointed at a sunspot and a magnetic pore for IBIS and ROSA, respectively.
    This can be seen in Figure \ref{fig:data_overview}.
    These two data sets are a good representative of the data sets studied within this thesis.
    
    The sigma (mean) value used came from a background box of quiet Sun, i.e., a region of the photosphere that contains no magnetic features.
    Once the sigma value is calculated, it is multiplied by a value which is called the sigma multiplier.  
    Both magnetic structures were contoured using a selection of sigma multipliers, 3, 3.5, 4, and 4.5 and 2, 2.5, 3, and 3.5 for the sunspot and magnetic pore respectively.
    These values correspond to the colours: blue, green, purple and orange.
    This contouring is displayed in Figure \ref{fig:method_overview}.
    The reason for difference in sigma multipliers between the sunspot and magnetic pore is due to the lack of a good quiet Sun region within the IBIS dataset.
    This is clearly seen in the top right histogram in Figure \ref{fig:method_overview}, where the returned distribution is skewed, and this had the result of increasing the sigma multipliers.
    
    The resultant cross-sectional area signals are analysed using the wavelet transform and this revealed a range of periods within these magnetic structures.
    For the IBIS sunspot, the range of sigma multipliers did not alter the periods found by the wavelet transform.
    What did vary was the wavelet power of the periods, as seen in Figure \ref{fig:sunspot_wavelet}. 
    The range of sigma multipliers return a threshold value that contours pixels which encompass the sunspot umbra and not the penumbra or background photosphere.
    This is why the sigma multiplier has little effect on the output in this case.
    For the ROSA magnetic pore, the difference in multipliers is more important.
    Figure \ref{fig:pore_wavelet} demonstrates that for the larger sigma multipliers, the smaller periods have disappeared from the wavelet transform. 
    This is because, the returned threshold value from the higher sigma multiplier under contours the magnetic pore.
    As a result, the returned cross-sectional signal is missing a large number of pixels, which can be seen by the signals at the top of Figure \ref{fig:pore_wavelet}.
    This could mean that only certain regions of the magnetic pore oscillate and might be possible to isolate those regions in a future work.
  
    The results can be summarised as the following.
    The best way to choose a sigma multiplier that will be used to create the threshold value, is to take into account the structures's intensity distribution.
    This way, choosing a sigma multiplier becomes more straightforward and more robust than choosing a threshold value as a percentage of the quiet Sun intensity.
    As long as the sigma multiplier is sane, then there will be nothing missed from the analysis.
    
    Finally, the phase difference or phase relations are used to identify the various waves modes and these are listed in Table \ref{tab:phase}.
    All three signal analysis methods offer the ability to calculate the phase of each signal and that allows the comparison of the cross-sectional area to the total intensity phase difference.
    For this analysis, the wavelet transform was used to check the phase difference between the cross-sectional area and total intensity.
    These can be seen in Figures \ref{fig:phase_sunspot} and \ref{fig:phase_pore} for the IBIS sunspot and ROSA magnetic pore, respectively.
    Overall, the periods show in phase behaviour within these two magnetic structures which indicates slow MHD sausage modes. 
    The cross-wavelet phase images show that there is no effect of the different sigma multipliers on wave mode identification.
    It should be noted that this study can be expanded upon which will be discussed later on. 
    
   	\subsection{Chapter 3}
    	
    Chapter 3 detailed the application of the method described previously in Chapter 2, to three magnetic structures: two sunspots and one magnetic pore.
    These are shown in Figure \ref{images}.
    These data sets came from two ground-based solar telescopes: the Dutch Open Telescope (DOT) and the Swedish Vacuum Solar Telescope (SVST). 
    These were described in Chapters 2 and 3 and while these telescopes are now out of service, they offered very decent data sets.
    
    Using a sigma multiplier of 2.5 to contour these magnetic structures, resulted in cross-sectional area and total intensity signals.
    Both the wavelet and EMD were employed to identify periods within these data sets and they revealed a range of periods within each magnetic structure's cross-sectional area and total intensity signals.
    These can be seen in Figures \ref{1999sunspot} and \ref{1999IMF}, \ref{2005sunspot} and \ref{20005IMF}, \ref{2008pore} and \ref{20008IMF}, for the sunspot observed with the SVST, the sunspot observed with the DOT and the magnetic pore observed with the DOT respectively.
    The periods found ranged from 2 to 40 minutes and many of the cross-sectional area periods had a corresponding total intensity period.
    Furthermore, a direct comparison of the cross-sectional area periods with line-of-sight (LOS) intensity oscillations found previously in sunspots demonstrates similar, if not the same, periods \cite{kobanov}.
    However, if they are linked or the same oscillation in a different form has yet to be established.
   
    The phase difference between the cross-sectional area and total intensity was calculated to be close to zero degrees, from both the wavelet and EMD analyses.
    Using the phase difference relations from Table \ref{tab:phase} indicates that the observed oscillations are slow MHD sausage waves. 
    This implies that there is a prevalent amount of slow MHD sausage waves in these magnetic structures which are located in the photosphere.
    In addition, there were small regions of out of phase and $\pm$45 degree behaviour.
    The out of phase behaviour indicates a fast MHD sausage waves, however this was not consistent and thus was ignored. 
    The $\pm$45 phase difference is more difficult to explain.
    While there is no current MHD theory that explains this phase difference, it has been shown that noise in a signal can cause the cross-wavelet phase to become shifted by $\pm$45 degrees \citep{2015A&A...579A..73M}.
    This would be one reason why the wavelet transform would need to be cross-checked with another signal analysis method.
    However, it should be noted that in \cite{2015A&A...579A..73M}, the signal to noise ratio is very low which is not generally the case with solar observations. 
    
    Finally, whether these oscillations are propagating or standing waves has and still is an open question. 
    Since it is not possible to distinguish between these propagating or standing waves using only the cross-sectional area and total intensity phase relations, it would require another observable.
    It has been previously suggested that these oscillations are standing oscillations.
    To this end, Table \ref{chap3:harm_table} lists the discovered periods and their corresponding period ratios, if these oscillations were harmonics.
    In order to calculate these period ratios, it was assumed that the largest period within that data set was the fundamental mode. 
    Furthermore, Table \ref{chap3:harm_table} lists the period ratios in the ideal homogeneous flux tube case.       
    Thus, by comparing these values to the observed period ratios, gives additional momentum, that these oscillations are standing harmonics; however, further investigation is required.
    
    \subsection{Chapter 4}
    	
    Chapter 4 expands on the previous work in Chapter 3 by studying two further magnetic pores.
    These two new data sets come from the DOT and the DST using the ROSA instrument, which offers an increase in resolution and a decrease in time cadence from the data sets studied in Chapter 3.
    These two data sets can be seen in Figure \ref{overview}.
    Again, the cross-sectional area and total intensity of each magnetic pore is calculated by the method previously used in Chapters 2 and 3. 
    Both magnetic pores display a collection of oscillations but due to the shorter length of these data sets, the maximum periods found were shorter.
    The periods range from 2 to 20 minutes and can be seen in Figures \ref{DOT_wls} and \ref{ROSA_wls}.     
    The phase difference between the cross-sectional area and total intensity signals show that these oscillations are slow MHD sausage waves.
    
    The extension within this chapter is utilisation of the perturbation amplitude of these oscillations.
    Using linear ideal MHD theory, it is possible to derive equations that will calculate the the ratio of magnetic field perturbation to the background magnetic field as well as the radial displacement and radial velocity perturbation of these oscillations.
    These are Equations \ref{eq:mag_area}, \ref{eq:area_rad}, and \ref{eq:rad_vel}, respectively. 
    To achieve this, the amplitude of the oscillations was required and the EMD was used to provide this.
    The IMFs returned from the EMD algorithm are known to return close to the actual amplitude of the original signal.
    To make sure this was accurate, these results were checked with the FFT and were within 10\% of each other.
	It should be noted that the wavelet transform can not be used to work out perturbation amplitudes.
	This is because the power spectrum is biased towards lower frequencies and thus must be normalised.
  
    For the DOT pore, the amplitudes for the cross-sectional area oscillations are $3.87\mathrm{x}10^5$, $3.61\mathrm{x}10^5$ and $5.90\mathrm{x}10^5$ km$^2$ for the oscillations with periods of 4.7, 8.5 and 20 minutes, respectively.
    The radial perturbation was calculated to be 37, 34, and 56 km and the radial velocity perturbation was calculated to be 0.82, 0.42, and 0.29 km s$^{-1}$.
    The obtained radial speeds are very sub-sonic, however, they are of order of observed horizontal flows around pores.
    Furthermore, the percentage change in the magnetic field was to be found 4-7\%\ which was found for another magnetic pore observed with DST/IBIS \citep{0004-637X-806-1-132}.
    To calculate the wavelength the phase speed of the wave needs to known.
    Due to previous MHD wave mode identification, the slow MHD sausage mode, the phase speed in a photospheric tube was calculated to be 5.2 km s$^{-1}$ using a background sunspot atmosphere model.
    Finally, the obtained estimate of the wavelength for these oscillations was $1269$ km, $2268$ km and $5319$ km.
       
	For the ROSA pore, the amplitudes for the cross-sectional area oscillations are $2.29\mathrm{x}10^5$, $2.45\mathrm{x}10^5$, and $3.87\mathrm{x}10^5$ km$^2$ for periods of 2-3, 5.5, and 10 minutes, respectively.
    The radial perturbation amplitude was calculated to be 69.1, 74.2, and 117 km and a radial velocity perturbation as 3.03, 1.41, and 1.23 km s$^{-1}$.
    The percentage change in the magnetic field was found to be  25-45\%.
    This change is several times the size of the DOT pore and should be measurable in future observations. 
    For this data set, there were no corresponding magnetograms and as a result, this effect could not be verified.
    These results suggests that the oscillation strength might be independent of the scale of the structure \citep{Dorotovic2014}. 
    Finally, the calculated wavelength was 540-810 km, 1485 km, and 2.2 Mm.
    A summary of these findings can be found in Tables \ref{tab:ampl} and \ref{tab:wavelength}.
      
    The calculated wavelengths are further evidence for standing harmonic oscillations within magnetic flux tubes in the photosphere.
    To show this, if the assumption that these are standing harmonics is taken, magneto-seismology can be used to prove or disprove this assumption.
    This is possible because magneto-seismology allows the calculation of two important background properties of magnetic flux tubes in this case.
    It should be noted that these flux tubes are photospheric flux tubes that start at the photosphere and end at the transition region.
    The first is density stratification, which is the ratio of the density at the top of the flux tube to the bottom of the flux tube.
    The second is the expansion factor ($\Gamma$), which is the ratio of the radius at the top of the flux tube to the bottom of the flux tube.
    These are Equations (\ref{den_strat}) and (\ref{mag_strat}) and the period ratio between the fundamental and first harmonic is the output value from these equations.
    Table \ref{harm_table} lists the period ratios of the observed oscillations for the two magnetic pores within this chapter.
    Furthermore, other period ratios are used from Chapter 3 for this analysis.

	For density stratification, three density models were used: VAL-III C, sunspot umbra and magnetic bright point.
	These came from \cite{1981ApJS...45..635V}, \cite{Maltby1986} and \cite{GFME13a,GFE14}, respectively.
	The resulting period ratio from these three models were 1.44, 1.38 and 1.41.
	This does not correspond well to the results presented in this chapter, but only for one previously reported result in Chapter 3.
	It can be concluded that density stratification does not seem to be applicable for these cases.

	For the expansion factor, Equation (\ref{mag_strat}) was solved for a range of expansion factors and plasma-$\beta$ values, which can be seen in Figure \ref{fig:harm}.
	For the results within this chapter, the flux tube has to expand four to six times to have a period ratio that is observed.
	This was compared to a number of numerical simulations that model these types of flux tubes and an observation of a coronal loop, which were in good agreement with these results, see \cite{khomenko,fedun2,fedun1} and \cite{2008A&A...489L..57K}.
	This creates a consistent image of standing harmonics that are supported between the photosphere and transition region within sunspots and magnetic pores.

	\subsection{Chapter 5}
    
    Chapter 5 shifts the focus from cross-sectional area analysis of sunspots and magnetic pores to the investigation of Running Penumbral Waves (RPWs).
    RPWs have been observed within sunspot penumbras since the 1970's as intensity fronts propagating radially outwards from the outer umbra into the penumbra, before disappearing at the penumbra photosphere boundary.
    To study RPWs, ground-based data from the Swedish Solar Telescope (SST) using the \textit{CRisp Imaging SpectroPolarimeter} (CRISP) instrument was combined with co-aligned and co-temporal data from the \textit{Atmospheric Imaging Assembly} (AIA) instrument on board the Solar Dynamics Observatory (SDO) satellite.
    An overview of the ground- and space-based data can be seen in Figure \ref{chap5:overview}.
    The focus was on a small Active Region (AR) containing two magnetic pores.
    The first magnetic pore had a light-bridge through the middle and it could not been seen in the chromosphere.
    The second magnetic pore was the focus for this observation and it was a larger magnetic pore that was seen clearly in the chromosphere. 
    Wideband and white light images from the SST and SDO, respectively, showed that these magnetic pores had no penumbral structure in the photosphere.
    
    By focusing on the chromosphere around the two magnetic pores, RPW-like events could be seen to emanate clearly from the larger magnetic pore.
    It was also noted that despite the smaller magnetic pore not being able to penetrate into the chromosphere, RPW-like waves could be seen to from the chromosphere above it. 
    To confirm if these were RPWs, a slit analysis was performed around the larger magnetic pore in order to find out the periodicity and speed of these RPWs.
    One example of the slit analysis can seen in Figure \ref{fft_slit} and the result of that slit analysis is as follows.
    The periodicity and speed of these RPW-like events are consistent with RPWs observed around sunspots.
    Furthermore, these events do not emanate concentrically around the magnetic pore as happens commonly with sunspots.
    The RPW-like events are confined to a small arc, the regions of the chromosphere which is not dominated with dynamic fibrils or large static fibrils, essentially the quiet chromosphere.
    It has been hypothesised that RPWs are an optical illusion caused by a delay in the appearance of MHD waves that travel along magnetic field lines and they have been termed Upwardly Propagating Waves (UPWs, \citealt{Bloomfiel2008}).
    If this was the case for the wave events observed here, understanding how the magnetic field behaviours around this magnetic pore is important. 
	Figure \ref{mag_field} shows the output of a magnetic field extrapolation code called MPole \citep{Longcope1996,Longcope2002}, which used magnetograms from \textit{Helioseismic and Magnetic Imager} (HMI) on board SDO as a source.
	It offers evidence that where the RPW-like events are observed, the magnetic field is significantly more radially inclined which is a requirement in order to observe RPWs if they are UPWs. 
	This result was the first direct imaging of RPWs in a magnetic pore in the H$\alpha$ line core and confirming that RPWs are in fact UPWs.
	 
	Furthermore, these UPWs are observed in two SDO/AIA (30.4 nm and 17.1 nm) lines that are formed in temperatures that correspond to the transition region and low corona.
	This suggests that UPWs are able to reach the hotter regions of the solar atmosphere, which is consistent as various MHD wave phenomena have been observed in the corona above sunspots previously. 
	Finally, it was found that the UPWs are most likely a superposition of MHD waves that have different dominant periods.
	This was found previously for RPWs observed around a sunspot \citep{Jess2013}. 
	
	As the understanding of RPWs increased, the identification of the wave type become important and the current literature indicates that these are slow magneto-acoustic waves \citep{Bloomfiel2008}.
	To identify the wave type for the UPWs observed in this chapter, the phase speed needs to be calculated.
	A time lag analysis between the H$\alpha$ slit and the SDO/AIA slits was attempted and returned a result of less than 12 seconds, i.e., the lag is less than the cadence of SDO/AIA.
	The best assumption that can be made is that the lag is at that cadence, 12 seconds.
	It should be noted that this lag could actually be zero seconds. 
	This would imply that H$\alpha$ and the two SDO/AIA lines have a temperature response, at least partially, within the same temperature range and thus the observation is of UPWs in three different wavelengths at the same time. 
	To confirm if this the case would require a future study with data set that consisted of ground-based telescope and multiple space-based telescope observations.
	
	Using the highest cadence will give the lower limit of the estimated phase speed, $42\pm21$ km s$^{-1}$.
	This speed is greater than the sound speed in the chromosphere which is estimated to be around 10 km s$^{-1}$ \citep{Morton2012} and is much closer to the Alfv\'en speed.
	Thus the phase speed indicates that this is a fast MHD wave and not a slow MHD wave that RPWs and UPWs are thought to be.
    This is the first reported observation that suggests that RPWs/UPWs are a fast MHD wave.
    
    Finally, with these UPWs having been identified within the MHD framework, it was possible to calculate the energy of the observed waves.
    Using an wave energy equation that uses the intensity and phase speed of the waves \citep{Kitagawa2010}, the energy is calculated to be around 150 W m$^{-2}$.
    This value is enough to supply the majority of the energy needed to heat the quiet Sun corona, however, it is a factor of 10 less than the energy required to heat the corona around an active region.
    It was previously found that the wave energy for cross-sectional area oscillations at lower chromospheric heights is of a similar value  \citep{0004-637X-806-1-132}. 
    However, when compared to other fast MHD wave energies measured within the chromosphere, it is a factor of 100 less \citep{Morton2012}. 
    Overall these waves play a small role in supplying energy to the corona, assuming a mechanism of dissipation, but they do help to reveal the nature of RPWs.
    
\section{Future work and questions}

	Within this thesis, many interesting MHD wave phenomena have been studied, discussed and understood.
    However, as a question is answered, further questions arise regarding these MHD waves.
    Here, I will detail the questions regarding the work I have presented and the work I will attempt in the future.
    This will be split into two parts: Methods and Science.
    
    \subsection{Methods}

	To begin, in Chapter 2 the signal analysis methods used within this thesis were described.
	These methods were used to analyse signals in order to measure the period and phase of a signal and they were used successfully to achieve the overall scientific goals for Chapters 3, 4, and 5.
	However, the field of signal analysis is ever evolving and many papers are published either suggesting improvements to current methods or offering new methods.
    The wavelet transform has had numerous papers published which extend the algorithm.
    For example, it was discovered that there is a power bias towards lower frequencies within the output \citep{liu2007rectification,veleda2012cross}.
    Furthermore, a modified version wavelet transform has been developed that is able to discern if the input signal is made up of standing or propagating components \citep{2008SoPh..248..395S}.
    The Empirical Mode Decomposition has been extended, adding a further step to the method where it does an ensemble averages to improve the outputted intrinsic mode functions (IMFs) \citep{wu2009ensemble}.
    Other ideas on how to deal with the edge effects with the spline fitting have been purposed \citep{zeng2004simple} and improved stopping criterion \citep{huang2008review}.
    Recently a new method has been created called Variational Mode Decomposition (VMD, \citealt{6655981}).
    It has a mathematical framework, much like the FFT and wavelet transform, that the EMD lacks and preliminary results suggest that it is more robust to sampling and noise effects than EMD.
    It is important that improved signal analysis methods are employed within solar physics as it more complex signals are interpreted within the literature, for example see \cite{refId0}. 
    Furthermore, the magneto-seismology equations used require the amplitude of the perturbation and as a result, any method that is more robust to noise or other signal analysis effects it is important that these be used as they give direct scientific answers.

    Finally, at the end of Chapter 2, was an analysis of the method used to measure the cross-sectional area and total intensity.
    To extend this study, it is important to look at the effect of light level change during an observation sequence as well as the effect if a pure intensity oscillation would result in a false detection of a cross-sectional area oscillation. 
    Theoretically, a pure intensity oscillation would result in a cross-sectional area oscillation due to a fixed sigma multiplier, as if it is brighter a larger cross-sectional area will be measured compared to when the intensity oscillation is dark and thus a smaller area would be measured.
    An artificial data set of a sunspot or a magnetic pore can be constructed to study these effects in the future.  
    Until then, how likely there will be a cross-talk between the cross-sectional area and total intensity signals is unknown.
                
    \subsection{Science}
 
	Within Chapter 2, it was shown that a high sigma multiplier would under estimate the cross-sectional area of a magnetic pore.
	The end result after signal analysis was that certain periods had vanished within that signal.
	This means that it might be possible to isolate the regions where the magnetic pore oscillates and could provide an insight into if the magnetic structure is  monolithic magnetic flux tube or if it is made up of a collection of smaller flux tubes.
 
    The focus within this thesis regarding how the cross-sectional area of sunspots and magnetic pore changes with time was within the photosphere only using a G-band filter.
    This filter samples the lower photosphere ($<$250 km).
    The first extension to this work would be utilising more filters that are found on ground-based telescope instruments.
    For example, \ion{Fe}{I} 630.26 nm, \ion{Na}{D} 589.70 nm, and \ion{Ca}{II} 854.16 nm which sample lower photosphere (\textasciitilde50 km), mid-photosphere (\textasciitilde 450 km) and mid-chromosphere (\textasciitilde1000 km), respectively.
    This would enable a study into how the cross-sectional area and total intensity vary as a function of height for a sunspot or magnetic pore. 
    The main issue with a study like this is that the boundary between the sunspot or magnetic pore and the background becomes less distinct in the higher formation lines.
    Thus the question of what is the actual structure becomes a difficult question to answer.
 
    Furthermore, atmospheric effects during an observational sequence can affect the quality of the data heavily and introduce artefacts \citep{2015A&A...579A..73M}.
    To counter act this, moving to space-based telescopes would remove the Earth's atmosphere.
    While SDO does not offer a high enough resolution to observe the cross-sectional area oscillations, Hinode as well as NASA's Interface Region Imaging Spectrograph (IRIS) telescope could be used to study these oscillations.
    Both satellites are able to observe the photosphere and wavelengths that sample the higher regions of the solar atmosphere. 
    Combining all of these into one comprehensive data set would enable the most detailed study of a sunspot or magnetic pore from the photosphere to the transition region and corona.  
    
    In Chapter 4 the phase speed of the observed MHD waves was calculated using typical background properties and not from any observational quantity.
    To measure the phase speed normally would require a multi-height analysis, as detailed above, to calculate the phase speed by calculating the time lag between the signal at different heights.
    Recently, \cite{2015A&A...579A..73M} detailed a theoretical MHD framework that would return a phase speed for any observed MHD wave by using the amplitude of the oscillations.
    This method can be verified by using observations. 
    Added to this, \cite{2015A&A...578A..60M} extended the previous MHD framework to calculate the energy of the MHD wave based off the phase speed and amplitude of the MHD wave.
    All this combined would allow a much deeper insight into MHD waves within sunspots and magnetic pores.

    The current range of ground-based and high-resolution space-based solar telescopes has approximately the same resolution and this limit will be around for another two years. 
    This puts a lower limit onto the size of perturbations measured for the cross-sectional area of sunspots and magnetic pores.
    There are two reasons why resolution is important.
    Firstly, as discussed in Chapter 1, the MHD surface modes for the slow MHD wave has a lower amplitude than the slow body mode and with current capabilities it is impossible to observe.
    Secondly, \citep{2015A&A...579A..73M} using an artificial sunspot data set detailed that due to sub-resolution perturbations, the thresholding routine did not correctly detect the contraction of the sunspot and that the magnitude of the cross-sectional area perturbation had been under estimated by a factor of 2.
	With Daniel K. Inouye Solar Telescope (DKIST) becoming operational within the next 5 years will significantly increase the resolution of solar data.
	Hopefully, Japan Aerospace Exploration Agency's (JAXA) Solar-C satellite and the European Solar telescope (EST) will join DKIST within the next decade.
	
	Finally, the observed UPWs within the magnetic pore, discussed in Chapter 5, offers an interesting avenue for MHD wave research.
	Here, there are many questions that need to be answered using an expanded data set.
	To begin, the observation indicated that these UPWs were fast sausage modes which is in opposition with the current literature.
	Their are two answers to this problem; either the analysis is incorrect or the interpretation is correct and UPWs can be both fast or slow MHD sausage waves.
	The second answer would lead to further insights as currently UPWs in sunspots are slow MHD sausage waves while in this magnetic pore they are fast MHD sausage waves.
	This could be due to the differing magnetic field geometries or background properties.
	For example, does the presence of the penumbra cause this difference or does the plasma-$\beta$ vary such that mode conversion leads to a slow MHD sausage wave instead of a fast MHD sausage wave?
	
	Furthermore, the region where the UPWs were observed was a small region of quiet chromosphere.
	Generally, symmetrical sunspots display concentric and clear RPWs, but is that due to the magnetic field geometry of the sunspot being much stronger than a magnetic pore?
	Thus do the observations of UPWs allow the ability to infer the magnetic field topology around sunspots and magnetic pores?
	This can be verified with a more complex magnetic field extrapolation code than the one used in Chapter 5.
	Finally, do UPWs have a common source with the LOS oscillations and cross-sectional area oscillations observed in these magnetic structures, or are they the same phenomena that is observed in several different ways due to the background plasma properties?
	Ground-based observations of sunspots and magnetic pores would need to be conducted in order to understand how UPWs vary within these structures.
	This can be coupled with full-Stokes polarimetric measurements of \ion{Ca}{II} $8542$ \AA\ spectral line and \ion{Fe}{I} $6302$ \AA\ which would allow the computation of the background density and temperature of the photosphere and chromosphere using NICOLE \citep{2015A&A...577A...7S,2015ApJ...798..100B}.
	
	While the Sun still offers more questions than answers; new analysis methods, faster simulations and higher resolution observations provide more ways to find answers to these questions.