\graphicspath{{Chapter6/Figs/}}

\chapter{Conclusion}
\label{chapter6}
    
    \vspace*{\fill}\par
    \pagebreak

\section{Overview of the thesis}
    
	In this thesis, the results of two-dimensional image analysis of sunspots and magnetic pores in the lower solar atmosphere is detailed.
    These results are directly compared to theoretically derived results and indicate the ubiquitous presence of slow MHD sausage waves in the larger magnetic structures that inhabit the solar surface.
    In Chapter 2, the telescopes and their associated instruments that supplied the data, that was analysed in this thesis are described.
    Then the signal analysis methods used to determine the periods and phase difference were detailed.
    Finally, the method used to measure the cross-sectional area and total intensity of the magnetic structures is examined.
    The reason for this is to understand what the effects of varying the sigma multiplier has on the results of the signal analysis, which is important in order to reduce bias.
    In Chapter 3, the cross-sectional area and total intensity is measured on several magnetic structures.
    The overall results indicate the presence of slow MHD sausage waves.
    This conclusion is reached by comparing the phase difference between the cross-sectional area and total intensity which was in phase.
    In Chapter 4, the cross-sectional area and total intensity is once again measured for two new magnetic pore data sets.
    The phase difference indicated strongly that the oscillations are slow MHD sausage waves.
    The extension within this chapter was calculating the amplitudes of the cross-sectional area perturbations.
    The amplitude of these perturbations allows the usage of magneto-seismology equations.
    These equations calculate several properties (radial) of the detected waves, as well as information on the background properties of the observed magnetic flux tubes.
    In Chapter 5, the focus shifted from analysing the cross-sectional of magnetic structures to the analysis of Running Penumbral Waves (RPWs) in a magnetic pore.
    RPWs have only observed in sunspot penumbras; however, this analysis details the presence of events that have the identical signature of RPWs, emanating from a magnetic pore.
    This fact has confirmed that RPWs are a visual effect of MHD sausage waves following the curvature of magnetic field lines of sunspots and magnetic pores.
    These have been previously termed Upwardly Propagating Waves (UPWs).
    This creates a delay of the appearance of the wave at higher levels, causing this patten of outwardly propagating waves.
    These events are able to deliver a small amount of energy into the local corona, but not enough to heat the active corona.
       
\section{Summary of results}

	\subsection{Chapter 2}

    Chapter 2 started off detailing the various solar telescopes, space- and ground-based, utilised within this thesis.
    It covered the instruments that was either on board the space-based telescopes or attached to an optics bench for ground-based telescopes.
    From here, the three signal analysis methods employed are described: Fast Fourier Transform (FFT), Wavelets and Empirical Mode Decomposition (EMD).
    
    The chapter finishes on a study of the principle method that is used to measure the cross-sectional area of the sunspots and magnetic pores analysed within this thesis.
    The motivation was to understand if the method output was heavily dependent or biased by the sigma multiplier value chosen.
    If this was the case, then this had to be known before this method could be used for any scientific analysis.    
    The sigma multiplier is a crucial factor as it will influence the end result, which is the cross-sectional area and total intensity signals.
    In order to undertake this study, high-quality ground-based data was used from the Dunn Solar Telescope (DST).
    The instruments used from the DST are the \textit{Interferometric Bidimensional Spectrometer} (IBIS) and the \textit{Rapid Oscillations in the Solar Atmosphere} (ROSA) instrument.
    There are two datasets, one for each instrument, with the telescope pointed at a sunspot and a magnetic pore for IBIS and ROSA, respectively.
    This can be seen in Figure \ref{fig:data_overview}.
    These two data sets are good representative of the data sets studied within this thesis.
    
    The sigma (mean) value used came from a background box of quiet Sun, i.e., a region of the photosphere that contains no magnetic features.
    Once the sigma value is calculated, it is multiplied by a value
    called the sigma multiplier.  
    Both magnetic structures were contoured using a selection of sigma multipliers, 3, 3.5, 4, and 4.5 and 2, 2.5, 3, and 3.5 for the sunspot and magnetic pore respectively.
    These values correspond to the colours: blue, green, purple and orange.
    This contouring is displayed in Figure \ref{fig:method_overview}.
    The reason for difference in sigma multipliers between the sunspot and magnetic pore is due to the lack of a good quiet Sun region within the IBIS dataset.
    This is clearly seen in the top right histogram in Figure \ref{fig:method_overview} and this had the result of increasing the sigma multipliers.
    
    The resultant cross-sectional area signals are analysed using the wavelet transform and this revealed a range of periods within these magnetic structures.
    For the IBIS sunspot, the range of sigma multipliers did not change the periods found by the wavelet transform.
    What did vary was the power of the periods in the wavelet is seen in Figure \ref{fig:sunspot_wavelet}. 
    The range of sigma multipliers return a threshold value that contours pixels which encompass the sunspot umbra and not the background photosphere.
    For the ROSA magnetic pore, the difference is more important.
    Figure \ref{fig:pore_wavelet} demonstrates that for the larger sigma multipliers, the smaller periods have disappeared from the wavelet transform. 
    This is because, the returned threshold value from the higher sigma multiplier under contours the magnetic pore.
    As a result, the returned cross-sectional signal is missing a large number of pixels, which can be seen by the signals at the top of Figure \ref{fig:pore_wavelet}.
        
    The results can be summarised as the following.
    The best way to choose a sigma multiplier that will be used to create the threshold value, is to take into account the structures's intensity distribution.
    This way, choosing a sigma multiplier becomes more straightforward and more robust than choosing a threshold value as a percentage of the quiet Sun intensity.
    As long as the sigma multiplier is sane, then there will be nothing missed from this analysis.
    
    Finally, since the phase relations are used to identified waves modes and this are listed in Table \ref{tab:phase}.
    All three signal analysis methods offer the ability to calculate the phase of each signal and that allows the comparison of the cross-sectional area to the total intensity phase.
    For this analysis, the wavelet transform was used to check the phase difference between the cross-sectional area and total intensity.
    These can be seen in Figures \ref{fig:phase_sunspot} and \ref{fig:phase_pore} for the IBIS sunspot and ROSA magnetic pore, respectively.
    Overall, the periods show in phase behaviour in these two magnetic structures indicating slow MHD sausage modes. 
    The cross-wavelet phase images show that there is no effect of the different sigma multipliers on wave identification. 
    
   	\subsection{Chapter 3}
    	
    Chapter 3 detailed the application of the method described and studied previously in Chapter 2 to three magnetic structures: two sunspots and one magnetic pore.
    These are shown in Figure \ref{images}.
    These date sets came from two ground-based solar telescopes: the Dutch Open Telescope (DOT) and the Swedish Vacuum Solar Telescope (SVST). 
    These were described in Chapters 2 and 3.
    While these telescopes are quite old and now out of service, they offered very decent data sets.
    
    Using a sigma multiplier of 2.5 to contour these magnetic structures
    resulted in cross-sectional area and total intensity signals.
    Both the wavelet and EMD were employed to identify periods within these data sets and they revealed a range of periods within each magnetic structure's cross-sectional area and total intensity.
    These can be seen in Figures \ref{1999sunspot}, \ref{1999IMF}, \ref{2005sunspot} and \ref{20005IMF}.
    The periods found ranged from 2 to 40 minutes and many of the cross-sectional area periods have a corresponding total intensity period.
    Furthermore, a direct comparison of the cross-sectional area periods with line-of-sight (LOS) intensity oscillations found previously in sunspots demonstrates similar, if not the same, periods.
    However, if they are linked or the same oscillation in a different form has not been established.
   
    The phase difference between the cross-sectional area and total intensity was calculated to be zero degrees, from both the wavelet and EMD analyses.
    Using the phase difference relations from Table \ref{tab:phase} indicates that the observed oscillations are slow MHD sausage waves. 
    This implies that there is a prevalent amount of slow MHD sausage waves in these magnetic structures which are located in the photosphere.
    In addition, there were small regions of, out of phase and $\pm$45 degrees.
    The out of phase phase difference indicates a fast MHD sausage waves, however this was not consistent and thus was ignored. 
    The $\pm$45 phase difference is more difficult to explain.
    While there is no current MHD theory that explains this phase difference, it has been shown that noise in a signal can cause the cross-wavelet phase to become shifted by $\pm$45 degrees \citep{2015A&A...579A..73M}.  
    
    Finally, whether these oscillations are propagating or standing waves is an open question.
    Since it is not possible to distinguish between these two using only the cross-sectional area and total intensity phase relations.
    This would require another observable.
    However, it has been previously suggested that these oscillations are standing oscillations.
    Table \ref{chap3:harm_table} lists the discovered periods and has the period ratios if these oscillations were harmonics.     
    The period ratios are calculated assuming the largest period is the fundamental.
    These are compared to the period ratios that would be supported in an ideal magnetic flux tube.
    This gives additional evidence that these oscillations are standing harmonics, however, further investigation is required.
    
    \subsection{Chapter 4}
    	
    Chapter 4 expands on the previous work from Chapter 3, by studying two further magnetic pores.
    These two new data sets come from the DOT and, the DST/ROSA instrument which offers an increase in resolution and a decrease in time cadence from the data sets studied in Chapter 3.
    These two data sets can be seen in Figure \ref{overview}.
    By once again, measuring the cross-sectional area and total intensity of each magnetic pore using the method previously in Chapters 2 and 3. 
    Both magnetic pores display a collection of oscillations but due to the shorter length of these data sets, the maximum periods found are shorter.
    The periods range from 2 to 20 minutes and can be seen in Figures \ref{DOT_wls} and \ref{ROSA_wls}.     
    The phase difference between the cross-sectional area and total intensity show that these oscillations are are slow MHD sausage waves.
    The extension within this chapter is utilisation of the perturbation amplitude of these oscillations.
    
    Using linear ideal MHD theory, it is possible to derive equations that will allow the calculation of the radial displacement speed of the oscillation as well as the magnetic field change of the structure due to the oscillation.
    To achieve this, the amplitude of the oscillations was required and the Fast Fourier Transform (FFT) was used to provide this.
    The radial displacement speeds were calculated to be within the range of 0.3 km s$^{-1}$ to 3 km s$^{-1}$. 
    Further, the magnetic field change was calculated to be from 5\% up 30\%. 
    These results are stated within Table \ref{tab:ampl}.
    The magnetic field change for the magnetic pore observed with ROSA is much larger than the magnetic pore observed in DOT.
    The reason is that the amplitude of the oscillations is of the same order for both magnetic pores despite that the ROSA magnetic pore has a much smaller cross-sectional area.
    Therefore, with the MHD wave type known, it is possible to estimate the phase speed for this wave type within a typical magnetic flux tube.
    This allows the calculation of the wavelength for the observed oscillations (see Table \ref{tab:wavelength}).
    The calculated wavelengths add more evidence to the anecdotal evidence shown in Chapter 3.
    There is a picture emerging that these oscillations could be standing harmonics supported within these magnetic flux tubes. 
    If the assumption that these are standing harmonics is taken, using magneto-seismology it is possible to estimate the expansion factor of the flux tubes.
    This is the ratio of the radius at the base of the flux tube to the radius at the top of the flux tube.
    In this case, it is from the photosphere to the transition region.
    Table \ref{harm_table} lists the period ratios of the observed oscillations for the two magnetic pores.
    This forms the base to calculate the expansion factor.
    The expansion factor is dependent on the plasma-$\beta$ at the base of the flux tube and while it is not possible to know this value, it can be assumed to be around 1 for flux tubes that are within the photosphere.
    The results of this can be seen within Figure \ref{fig:harm}, which gives a factor that ranges from 4-8 for the magnetic structures analysed within this thesis.
    These numbers are not too dissimilar to the expansion factors for flux tubes used within MHD wave simulations.

	\subsection{Chapter 5}
    
    In Chapter 5, the focus shifts from the analysis of the cross-sectional area of magnetic structures to Running Penumbral Waves (RPWs).
    RPWs have been observed in sunspots since the 1970's as intensity fronts propagating radially outwards from the outer umbra into the penumbra.
    Excellent seeing data from the Swedish Solar Telescope (SST) using the \textit{CRisp Imaging SpectroPolarimeter} (CRISP) instrument was combined with co-aligned and co-temporal data from the \textit{Atmospheric Imaging Assembly} (AIA) instrument onboard the Solar Dynamics Observatory (SDO) satellite (see Figure \ref{chap5:overview}).
    The result was the first direct imaging of RPWs in a magnetic pore in the H$\alpha$ line core.
    The whitelight images show no penumbral structure in the photosphere for the observed magnetic pore.
    Further, the repeat period as well as the horizontal speed is typical of RPWs that have been observed before within sunspots.
    The RPWs can be seen to emanate from the magnetic pore radially outwards, however, it is not concentric as RPWs that are commonly observed within sunspots.
    These results came from slit analysis around the magnetic pore (see Figure \ref{chap5:overview} and \ref{fft_slit}).
    The RPWS are confined to a small region, which is a quiet part of the chromosphere.
    The waves do not appear to propagate in the region with large static fibrils and regions with dynamic fibrils.
    The answer from a magnetic field extrapolation (Figure \ref{mag_field}) which suggests that the field where the RPWs are observed is more radial (i.e., horizontal to the surface) than other parts of the magnetic field.
    This fact coupled with previous research that indicates that RPWs are in fact Upwardly Propagating Waves (UPWs), is the final step in confirming this.     
    Further, the UPWs were observed with SDO/AIA lines that sample the transition region and low corona, which suggests that UPWs are reaching the higher parts of the solar atmosphere.
    Mode identification for RPWs has been a difficult topic. 
    Previous studies and most theoretical understanding of RPWs implies that they are slow magneto-acoustic waves.
    However, in the case presented within Chapter 5, we suggest that the RPWs/UPWs are actually fast magneto-acoustic waves, since the wave appears as dark and light fronts, which is a change in intensity which we conclude is a change in density implying a compressive nature to this waves.
    A time lag analysis between the H$\alpha$ slit and the SDO/AIA slits gives us a result of less than 12 seconds, i.e., the lag is less than the cadence of SDO/AIA.
    So the best assumption that can be made is that the lag is at that cadence (12 seconds). 
    This gives the low phase speed estimate, but the result suggests that the UPW phase speed is greater than the sound and Alfv\'en speed (assuming typical chromospheric physical values) which means it is a fast wave. 
    However, this is the first report that of a RPW/UPW as a fast mode.
    With the wave mode identified, the next was the calculation of the energy of the observed waves.
    The energy is calculated to be around 150 W m$^{-2}$, which is enough to heat the quiet Sun corona but not an active region corona.
    This value has been found by another study for cross-sectional area oscillations in the low chromosphere.
    So overall, these waves have a small but important contribution to corona heating.
    
\section{Future work and questions}
    
    
        For future, does this isolate the regions where the pore oscillations? 
        
      
      Further, this implies that for the current range of ground-based solar telescopes, that the limiting factor for cross-sectional area oscillations is the resolution.
      This resolution has been fixed for the past couple of years and until Daniel K. Inouye Solar Telescope (DKIST) data becomes available, the range of sigma multipliers can vary without fear of biasing the results. 
          
          
          potential cross talk between area and intensity is ignored. Pure intensity oscillation would result in area oscillation due to a fixed sigma threshold. brighter > larger area ; darker > smaller area
          
             
    This is always a tricky subject.
    With any body of work, there are always unanswered questions or gaps in knowledge that ideally should be filled.
    As such, this thesis does not offer a full view of MHD sausage waves in sunspots and magnetic pores.
    
    Firstly, in Chapter 2, the method used to contour the magnetic structures was analysed.
    However, it was limited to low lying photospheric wavelengths.
    This would be expanded to cover other wavelengths which sample up into the chromosphere. 
    Studying how the cross-sectional area oscillations change as a function of height would be an important next stage for this research.
    In order to map how the amplitudes change with height, if the phase difference varies and if detected periods at one height exist within another height.
    However, as one moves upwards in the solar atmosphere, the boundary between a magnetic structure and the background atmosphere becomes harder to distinguish, so the analysis in Chapter 2 would need to be refined with multiple heights to investigate any pitfalls with layered datasets.
    Further to this, the current range of ground-based and space-based solar data has approximately the same resolution and this limit will be around for another two years. 
    This is a problem since the results are inherently dependant on this as this fixes the lower limit of any observations, especially when it comes to detecting perturbations.
    With DKIST, as well as Solar-C, this limit will be reduced.
    Thus, attempting this analysis on solar data from the next generation of solar telescopes would be a future extension of this work. 
    Finally, in Chapter 4, the phase speed of the MHD wave was calculated from typical background properties.
    Thus, the results that came from that are determined by those properties. 
    Further work would ideally employ magneto-seismology in order to calculate the phase speed of the detected oscillations. 
    Normally, this is possible using multiple heights since a time lag could be calculated with this method, as the height between wavelengths is known approximately.
    However, research \citep{2015A&A...579A..73M} have been published that makes it possible to do this with only the amplitude of the oscillations.
    This would be a further extension to the work in Chapter 3 and 4.
    
    Secondly, Chapter 2 detailed the signal analysis methods employed within this thesis.
    While these methods were useful and achieved the overall goals for each chapter, the field of signal analysis is ever evolving.
    For example, the wavelet transform has had numerous papers published which extend the algorithm to account for power bias at lower periods \citep{liu2007rectification,veleda2012cross} as well as being able to discern if the signal is standing or propagating \citep{2008SoPh..248..395S}.
    The Empirical Mode Decomposition has been extended, adding a step to the method where it does an ensemble averages \citep{wu2009ensemble}, better methods to deal with the edge effects with the spline fit \citep{zeng2004simple} as well as improved stopping criterion \citep{huang2008review}.
    It is important that the methods used to analyse signals and extract phase relations and amplitudes be made robuster in order to be certain in the period, phase and amplitudes of any oscillations found.
    If magneto-seismology requires these quantities to calculate the background properties of flux tubes, it is vital that any signal analysis method used is as robust as possible. 
    
    Finally, the observed UPWs within the magnetic pore is an interesting avenue of wave research.
    Here, there are many questions that need to be answered.
    To start, this observation indicated that the UPW events were fast sausage modes needs to be squared with the current literature.
    Either the interpretation is correct and UPWs can be either fast or slow waves or the analysis is incorrect.
    Further ground-based observations of magnetic pores would need to be conducted in order to understand UPWs within these structures.
    The author is only aware that RPWs have been only observed in sunspots , is the difference in wave mode due to a difference between sunspots and magnetic pores? 
    Does the presence of the penumbra cause this or would it be that the plasma-$\beta$ varies in a pore differently such that mode conversion leads to a fast magnetic-acoustic wave instead of a slow magnetic-acoustic wave?
    In Chapter 5, the region where the waves were observed was a small region of quiet chromosphere.
    Does the surrounding chromospheric atmosphere dictate if UPWs can be observed?
    Generally, symmetrical sunspots display concentric and clear RPWs, is that due to the magnetic field topology of the sunspot being much stronger than a magnetic pore, thus the observation of UPWs allows us to infer the magnetic field topology around sunspots and magnetic pores?
    Finally, do UPWs have a common source with the LOS oscillations and cross-sectional area oscillations observed in these magnetic structures, or are they the same wave that can be observed in several different ways?
    Solar physics has a long way to go and so does the research presented within this Thesis.