\graphicspath{{Chapter6/Figs/}}

\chapter{Conclusion}
\label{chapter6}
    
    \vspace*{\fill}\par
    \pagebreak

\section{Overview of the thesis}
    
	In this thesis, the results of two-dimensional image analysis of sunspots and magnetic pores in the lower solar atmosphere is detailed.
    These results are directly compared to theoretically derived results and indicate the ubiquitous presence of slow MHD sausage waves in the larger magnetic structures that inhabit the solar surface.
    
    In Chapter 2, the telescopes and their associated instruments that supplied the science ready data sets are described, along with brief data reduction details.
    Then the signal analysis methods used to determine the periods and phase difference were detailed.
    These being the Fast Fourier Transform (FFT), Wavelets and Empirical Mode Decomposition (EMD).
    Finally, the method used to measure the cross-sectional area and total intensity of the magnetic structures is examined.
    The reason for this is to understand what the effects of varying the sigma multiplier has on the results of the signal analysis, which is important before this method is used to analyse data sets.
    
    In Chapter 3, the cross-sectional area and total intensity is measured on two sunspots and one magnetic pore.
    By comparing the phase difference between the cross-sectional area and total intensity signals, which are calculated using the method described in Chapter 2.
    It was possible to find the ubiquitous presence of slow MHD sausage waves within these magnetic structures.
    This conclusion is reached because the measured phase difference was very close to 0$^\circ$, i.e., they were in phase.
    This is the signature of the slow MHD sausage wave within cylindrical magnetic flux tubes.
    
    In Chapter 4, the cross-sectional area and total intensity is calculated for two new magnetic pore data sets.
    The phase difference indicated that the oscillations are slow MHD sausage waves.
    The usage of magneto-seismology equations allowed the calculation of several properties of the detected oscillations within the two magnetic pores.
    These were, for example, the radial distance perturbation, radial velocity perturbation and magnetic field perturbations.
    The properties of these oscillations gave the impression of standing harmonics.
    However, it was possible to use magneto-seismology to demonstrate that these oscillations are in fact standing harmonics.
    This was accomplished by working out if density stratification or radial expansion of the flux can cause the observed period ratios.
    The calculated radial expansion was in good agreement with previous simulations and observations. 
    
    In Chapter 5, the focus shifted from analysing the cross-sectional of magnetic structures to the analysis of Running Penumbral Waves (RPWs) in a magnetic pore.
    RPWs have only observed in sunspot penumbras; however, RPW-like events were observed, for the first time, to occur around a magnetic pore.
    This observation is the final step in confirming that RPWs are a visual effect of MHD sausage waves following the curvature of magnetic field lines of sunspots and magnetic pores.
    These have been previously termed Upwardly Propagating Waves (UPWs).
    This visual effect is caused because the magnetic field becomes more radially inclined further from the umbral centre.
    These radially inclined fields have longer arc lengths and thus a wave following these field lines have to cover a larger distance before it can appear in the chromosphere.
    This creates a delay of the appearance of these waves at higher levels, causing this patten of radially outward propagating waves.
    These events are able to deliver a small amount of energy into the local corona, but not enough to heat the active corona.
       
\section{Summary of results}

	\subsection{Chapter 2}

    Chapter 2 started by detailing the various solar telescopes, space- and ground-based, utilised within this thesis.
    It covered the instruments that was either on board the space-based telescopes or attached to an optics bench for ground-based telescopes.
    From here, the three signal analysis methods employed are described: FFT, Wavelets and EMD.
    It detailed how each method works, strengths and weakness and why three methods were used instead of just one.
    
    The chapter finishes on a study of the principle method that is used to measure the cross-sectional area of the sunspots and magnetic pores analysed within this thesis.
    The motivation was to understand if the method output was heavily affected or dependent on the sigma multiplier value chosen.
    If this was the case, then this had to be known before this method could be used for any scientific analysis.    
    The sigma multiplier is a crucial factor as it will influence the end result, which is the cross-sectional area and total intensity signals.
    In order to undertake this study, high-quality ground-based data was used from the Dunn Solar Telescope (DST).
    The instruments used from the DST are the \textit{Interferometric Bidimensional Spectrometer} (IBIS) and the \textit{Rapid Oscillations in the Solar Atmosphere} (ROSA) instrument.
    There are two datasets, one for each instrument, with the telescope pointed at a sunspot and a magnetic pore for IBIS and ROSA, respectively.
    This can be seen in Figure \ref{fig:data_overview}.
    These two data sets are a good representative of the data sets studied within this thesis.
    
    The sigma (mean) value used came from a background box of quiet Sun, i.e., a region of the photosphere that contains no magnetic features.
    Once the sigma value is calculated, it is multiplied by a value which is called the sigma multiplier.  
    Both magnetic structures were contoured using a selection of sigma multipliers, 3, 3.5, 4, and 4.5 and 2, 2.5, 3, and 3.5 for the sunspot and magnetic pore respectively.
    These values correspond to the colours: blue, green, purple and orange.
    This contouring is displayed in Figure \ref{fig:method_overview}.
    The reason for difference in sigma multipliers between the sunspot and magnetic pore is due to the lack of a good quiet Sun region within the IBIS dataset.
    This is clearly seen in the top right histogram in Figure \ref{fig:method_overview}, where the returned distribution is skewed, and this had the result of increasing the sigma multipliers.
    
    The resultant cross-sectional area signals are analysed using the wavelet transform and this revealed a range of periods within these magnetic structures.
    For the IBIS sunspot, the range of sigma multipliers did not alter the periods found by the wavelet transform.
    What did vary was the wavelet power of the periods, as seen in Figure \ref{fig:sunspot_wavelet}. 
    The range of sigma multipliers return a threshold value that contours pixels which encompass the sunspot umbra and not the penumbra or background photosphere.
    This is why the sigma multiplier has little effect on the output in this case.
    For the ROSA magnetic pore, the difference in multipliers is more important.
    Figure \ref{fig:pore_wavelet} demonstrates that for the larger sigma multipliers, the smaller periods have disappeared from the wavelet transform. 
    This is because, the returned threshold value from the higher sigma multiplier under contours the magnetic pore.
    As a result, the returned cross-sectional signal is missing a large number of pixels, which can be seen by the signals at the top of Figure \ref{fig:pore_wavelet}.
    This could mean that only certain regions of the magnetic pore oscillate and might be possible to isolate those regions in a future work.
  
    The results can be summarised as the following.
    The best way to choose a sigma multiplier that will be used to create the threshold value, is to take into account the structures's intensity distribution.
    This way, choosing a sigma multiplier becomes more straightforward and more robust than choosing a threshold value as a percentage of the quiet Sun intensity.
    As long as the sigma multiplier is sane, then there will be nothing missed from the analysis.
    
    Finally, the phase difference or phase relations are used to identify the various waves modes and these are listed in Table \ref{tab:phase}.
    All three signal analysis methods offer the ability to calculate the phase of each signal and that allows the comparison of the cross-sectional area to the total intensity phase difference.
    For this analysis, the wavelet transform was used to check the phase difference between the cross-sectional area and total intensity.
    These can be seen in Figures \ref{fig:phase_sunspot} and \ref{fig:phase_pore} for the IBIS sunspot and ROSA magnetic pore, respectively.
    Overall, the periods show in phase behaviour within these two magnetic structures which indicates slow MHD sausage modes. 
    The cross-wavelet phase images show that there is no effect of the different sigma multipliers on wave mode identification.
    It should be noted that this study can be expanded upon which will be discussed later on. 
    
   	\subsection{Chapter 3}
    	
    Chapter 3 detailed the application of the method described previously in Chapter 2, to three magnetic structures: two sunspots and one magnetic pore.
    These are shown in Figure \ref{images}.
    These data sets came from two ground-based solar telescopes: the Dutch Open Telescope (DOT) and the Swedish Vacuum Solar Telescope (SVST). 
    These were described in Chapters 2 and 3.
    While these telescopes are now out of service, they offered very decent data sets.
    
    Using a sigma multiplier of 2.5 to contour these magnetic structures, resulted in cross-sectional area and total intensity signals.
    Both the wavelet and EMD were employed to identify periods within these data sets and they revealed a range of periods within each magnetic structure's cross-sectional area and total intensity signals.
    These can be seen in Figures \ref{1999sunspot} and \ref{1999IMF}, \ref{2005sunspot} and \ref{20005IMF}, \ref{2008pore} and \ref{20008IMF}, for the sunspot observed with the SVST, the sunspot observed with the DOT and the magnetic pore observed with the DOT respectively.
    The periods found ranged from 2 to 40 minutes and many of the cross-sectional area periods had a corresponding total intensity period.
    Furthermore, a direct comparison of the cross-sectional area periods with line-of-sight (LOS) intensity oscillations found previously in sunspots demonstrates similar, if not the same, periods \cite{kobanov}.
    However, if they are linked or the same oscillation in a different form has yet to be established.
   
    The phase difference between the cross-sectional area and total intensity was calculated to be close to zero degrees, from both the wavelet and EMD analyses.
    Using the phase difference relations from Table \ref{tab:phase} indicates that the observed oscillations are slow MHD sausage waves. 
    This implies that there is a prevalent amount of slow MHD sausage waves in these magnetic structures which are located in the photosphere.
    In addition, there were small regions of out of phase and $\pm$45 degree behaviour.
    The out of phase behaviour indicates a fast MHD sausage waves, however this was not consistent and thus was ignored. 
    The $\pm$45 phase difference is more difficult to explain.
    While there is no current MHD theory that explains this phase difference, it has been shown that noise in a signal can cause the cross-wavelet phase to become shifted by $\pm$45 degrees \citep{2015A&A...579A..73M}.
    This would be one reason why the wavelet transform would need to be cross-checked with another signal analysis method.
    However, it should be noted that in \cite{2015A&A...579A..73M}, the signal to noise ratio is very low which is not generally the case with solar observations. 
    
    Finally, whether these oscillations are propagating or standing waves has and still is an open question. 
    Since it is not possible to distinguish between these propagating or standing waves using only the cross-sectional area and total intensity phase relations, it would require another observable.
    It has been previously suggested that these oscillations are standing oscillations.
    To this end, Table \ref{chap3:harm_table} lists the discovered periods and their corresponding period ratios, if these oscillations were harmonics.
    In order to calculate these period ratios, it was assumed that the largest period within that data set was the fundamental mode. 
    Furthermore, Table \ref{chap3:harm_table} lists the period ratios in the ideal homogeneous flux tube case.       
    Thus, by comparing these values to the observed period ratios, gives additional momentum, that these oscillations are standing harmonics; however, further investigation is required.
    
    \subsection{Chapter 4}
    	
    Chapter 4 expands on the previous work from Chapter 3, by studying two further magnetic pores.
    These two new data sets come from the DOT and, the DST/ROSA instrument which offers an increase in resolution and a decrease in time cadence from the data sets studied in Chapter 3.
    These two data sets can be seen in Figure \ref{overview}.
    By once again, measuring the cross-sectional area and total intensity of each magnetic pore using the method previously used in Chapters 2 and 3. 
    Both magnetic pores display a collection of oscillations but due to the shorter length of these data sets, the maximum periods found are shorter.
    The periods range from 2 to 20 minutes and can be seen in Figures \ref{DOT_wls} and \ref{ROSA_wls}.     
    The phase difference between the cross-sectional area and total intensity show that these oscillations are slow MHD sausage waves.
    
    The extension within this chapter is utilisation of the perturbation amplitude of these oscillations.
    Using linear ideal MHD theory, it is possible to derive equations that will calculate the the ratio of magnetic field perturbation to the background magnetic field due to the oscillation as well as the radial displacement and radial velocity perturbation of the oscillation.
    These are Equations \ref{eq:mag_area}, \ref{eq:area_rad}, and \ref{eq:rad_vel}, respectively. 
    To achieve this, the amplitude of the oscillations was required and the EMD was used to provide this.
    The IMFs returned from the EMD algorithm are known to return close to the actual amplitude of the signal.
    To make sure this was correct, these results were checked with the FFT were within 10\% of each other.
	It should be noted that the wavelet transform can not be used to work out perturbation amplitudes.
	This is because the power spectrum is biased towards lower frequencies and thus must be normalised.
  
    For the DOT pore, the amplitudes for the cross-sectional area oscillations are $3.87\mathrm{x}10^5$, $3.61\mathrm{x}10^5$ and $5.90\mathrm{x}10^5$ km$^2$ for the oscillations with periods of 4.7, 8.5 and 20 minutes, respectively.
    The area perturbation was calculated to be 37, 34, and 56 km and the radial velocity perturbation was calculated to be 0.82, 0.42, and 0.29 km s$^{-1}$.
    The obtained radial speeds are very sub-sonic, however, they are of order of observed horizontal flows around pores.
    Furthermore, the percentage change in the magnetic field was to be found 4-7\%\ which was found for another magnetic pore observed with DST/IBIS \citep{0004-637X-806-1-132}.
    To calculate the wavelength the phase speed of the wave needs to known.
    Due to previous MHD wave mode identification, the result was the slow MHD sausage mode, the phase speed in a photospheric tube was calculated to be 5.2 km s$^{-1}$.
    This was using a background sunspot atmospheric model.
    Finally, the obtained estimates of the wavelength (wavenumber) for these oscillation was $1269$ km ($4.95\mathrm{x}10^{-6}$ m$^{-1}$), $2268$ km ($2.77\mathrm{x}10^{-6}$ m$^{-1}$) and $5319$ km ($1.18\mathrm{x}10^{-6}$ m$^{-1}$).
       
	For the ROSA pore, the amplitudes for the cross-sectional area oscillations are $2.29\mathrm{x}10^5$, $2.45\mathrm{x}10^5$, and $3.87\mathrm{x}10^5$ km$^2$ for periods of 2-3, 5.5, and 10 minutes, respectively.
    The radial perturbation amplitude was calculated to be 69.1, 74.2, and 117 km and a radial velocity perturbation as 3.03, 1.41, and 1.23 km s$^{-1}$.
    The percentage change in the magnetic field was found to be  25-45\%.
    This changes is several times the size of the DOT pore and should be measurable in future observations. 
    For this data set, there were no corresponding magnetograms and as a result, could not be verified.
    These results suggests that the oscillation strength might be independent of the scale of the structure \citep{Dorotovic2014}. 
    Finally, the calculated wavelength (wavenumber) was $540$-$810$ km ($7.76\mathrm{x}10^{-6}$ m$^{-1}$), 1485 km ($3.58\mathrm{x}10^{-6}$ m$^{-1}$), and 2.2 Mm ($2.85\mathrm{x}10^{-6}$ m$^{-1}$).
    A summary of these findings can be found in Tables \ref{tab:ampl} and \ref{tab:wavelength}.
      
    The calculated wavelengths are further evidence for standing harmonic oscillations within magnetic flux tubes in the photosphere.
    To show this, if the assumption that these are standing harmonics is taken, using magneto-seismology can prove or disprove this assumption.
    This is possible because magneto-seismology allows the calculation of two important properties of flux tubes.
    It should be noted that these flux tubes are photospheric flux tubes that start at the photosphere and end at the transition region.
    The first is density stratification, which is the ratio of the density at the base of the flux tube to the top of the flux tube.
    The second is the expansion factor ($\Gamma$), which is the ratio of the radius at the base of the flux tube to the top of the flux tube.
    These are Equations (\ref{den_strat}) and (\ref{mag_strat}) and the period ratio between the fundamental and first harmonic is the output value.
    Table \ref{harm_table} lists the period ratios of the observed oscillations for the two magnetic pores and other period ratios are used from Chapter 3.

	For density stratification three density models were used: VAL-III C, sunspot umbra and magnetic bright point.
	These came from \cite{1981ApJS...45..635V}, \cite{Maltby1986} and \cite{GFME13a,GFE14}, respectively.
	The resulting period ratio from these three models were 1.44, 1.38 and 1.41.
	This does not correspond well to the results presented in this chapter, but only for one previously reported result in Chapter 3.
	It can be concluded that density stratification does not seem to be applicable in this case.

	For the expansion factor, Equation (\ref{mag_strat}) was solved for a range of expansion factors and plasma-$\beta$ values, which can be seen in Figure \ref{fig:harm}.
	Over-plotted in this parameter space are the period ratios observed in blue.
	For the results within this chapter, the flux tube has to expand four to six times to have a period ratio that is observed.
	This can be compared to a number of numerical simulations that model these types of flux tubes and an observation of a coronal loop which were in good agreement with these results, see \cite{khomenko,fedun2,fedun1} and \cite{2008A&A...489L..57K}.
	This creates a consistent image of standing harmonics that are supported between the photosphere and transition region within sunspots and magnetic pores.

	\subsection{Chapter 5}
    
    Chapter 5 shifts the focus from cross-sectional area analysis of sunspots and magnetic pores to the investigation of Running Penumbral Waves (RPWs).
    RPWs have been observed within sunspot penumbras since the 1970's as intensity fronts propagating radially outwards from the outer umbra into the penumbra, before disappearing at the penumbra photosphere boundary.
    Data from the Swedish Solar Telescope (SST) using the \textit{CRisp Imaging SpectroPolarimeter} (CRISP) instrument was combined with co-aligned and co-temporal data from the \textit{Atmospheric Imaging Assembly} (AIA) instrument on board the Solar Dynamics Observatory (SDO) satellite.
    An overview of the ground- and space-based data can be seen in Figure \ref{chap5:overview}.
    The focus was on a small Active Region (AR) containing two magnetic pores.
    The first one was a small magnetic pore that had a light-bridge which did not penetrate into the chromosphere.
    The second one, which was the focus for this observation, was a larger magnetic pore that was seen clearly in the chromosphere. 
    Wideband and white light images from the SST and SDO, respectively, showed that these magnetic pores had no penumbral structure in the photosphere.
    
    
    The result was the first direct imaging of RPWs in a magnetic pore in the H$\alpha$ line core.
    Further, the repeat period as well as the horizontal speed is typical of RPWs that have been observed before within sunspots.
    The RPWs can be seen to emanate from the magnetic pore radially outwards, however, it is not concentric as RPWs that are commonly observed within sunspots.
    These results came from slit analysis around the magnetic pore (see Figure \ref{chap5:overview} and \ref{fft_slit}).
    The RPWS are confined to a small region, which is a quiet part of the chromosphere.
    The waves do not appear to propagate in the region with large static fibrils and regions with dynamic fibrils.
    The answer from a magnetic field extrapolation (Figure \ref{mag_field}) which suggests that the field where the RPWs are observed is more radial (i.e., horizontal to the surface) than other parts of the magnetic field.
    This fact coupled with previous research that indicates that RPWs are in fact Upwardly Propagating Waves (UPWs), is the final step in confirming this.     
    Further, the UPWs were observed with SDO/AIA lines that sample the transition region and low corona, which suggests that UPWs are reaching the higher parts of the solar atmosphere.
    Mode identification for RPWs has been a difficult topic. 
    Previous studies and mosWt theoretical understanding of RPWs implies that they are slow magneto-acoustic waves.
    However, in the case presented within Chapter 5, we suggest that the RPWs/UPWs are actually fast magneto-acoustic waves, since the wave appears as dark and light fronts, which is a change in intensity which we conclude is a change in density implying a compressive nature to this waves.
    A time lag analysis between the H$\alpha$ slit and the SDO/AIA slits gives us a result of less than 12 seconds, i.e., the lag is less than the cadence of SDO/AIA.
    So the best assumption that can be made is that the lag is at that cadence (12 seconds). could be zero!
    This gives the low phase speed estimate, but the result suggests that the UPW phase speed is greater than the sound and Alfv\'en speed (assuming typical chromospheric physical values) which means it is a fast wave. 
    However, this is the first report that of a RPW/UPW as a fast mode.
    With the wave mode identified, the next was the calculation of the energy of the observed waves.
    The energy is calculated to be around 150 W m$^{-2}$, which is enough to heat the quiet Sun corona but not an active region corona.
    This value has been found by another study for cross-sectional area oscillations in the low chromosphere.
    So overall, these waves have a small but important contribution to corona heating.
    
\section{Future work and questions}
    
    
        For future, does this isolate the regions where the pore oscillations? 
        
      
      Further, this implies that for the current range of ground-based solar telescopes, that the limiting factor for cross-sectional area oscillations is the resolution.
      This resolution has been fixed for the past couple of years and until Daniel K. Inouye Solar Telescope (DKIST) data becomes available, the range of sigma multipliers can vary without fear of biasing the results. 
          
          
          potential cross talk between area and intensity is ignored. Pure intensity oscillation would result in area oscillation due to a fixed sigma threshold. brighter > larger area ; darker > smaller area
   
   why is this the case? What bias can be introduced due to lack of resolution?
   an area variation is measured - area is defined by contouring intensity - intensity variation is introduced by some waves - the very thing we are trying to measure - is there potential for cross-talk, how addressed?
   
   again - is there a possibility that the intensity variation leads to the "appearance of cross sectional variation.
   
   still not clear how resolution limits or influences the results. What would we expect to gain or see that we don't see now?
   
          
             
    With any body of work, there are always unanswered questions or gaps in knowledge that ideally should be filled.
    As such, this thesis does not offer a full view of MHD sausage waves in sunspots and magnetic pores.
    
    Firstly, in Chapter 2, the method used to contour the magnetic structures was analysed.
    However, it was limited to low lying photospheric wavelengths.
    This would be expanded to cover other wavelengths which sample up into the chromosphere. 
    Studying how the cross-sectional area oscillations change as a function of height would be an important next stage for this research.
    In order to map how the amplitudes change with height, if the phase difference varies and if detected periods at one height exist within another height.
    However, as one moves upwards in the solar atmosphere, the boundary between a magnetic structure and the background atmosphere becomes harder to distinguish, so the analysis in Chapter 2 would need to be refined with multiple heights to investigate any pitfalls with layered datasets.
    Further to this, the current range of ground-based and space-based solar data has approximately the same resolution and this limit will be around for another two years. 
    This is a problem since the results are inherently dependant on this as this fixes the lower limit of any observations, especially when it comes to detecting perturbations.
    With DKIST, as well as Solar-C, this limit will be reduced.
    Thus, attempting this analysis on solar data from the next generation of solar telescopes would be a future extension of this work. 
    Finally, in Chapter 4, the phase speed of the MHD wave was calculated from typical background properties.
    Thus, the results that came from that are determined by those properties. 
    Further work would ideally employ magneto-seismology in order to calculate the phase speed of the detected oscillations. 
    Normally, this is possible using multiple heights since a time lag could be calculated with this method, as the height between wavelengths is known approximately.
    However, research \citep{2015A&A...579A..73M} have been published that makes it possible to do this with only the amplitude of the oscillations.
    This would be a further extension to the work in Chapter 3 and 4.
    
    Secondly, Chapter 2 detailed the signal analysis methods employed within this thesis.
    While these methods were useful and achieved the overall goals for each chapter, the field of signal analysis is ever evolving.
    For example, the wavelet transform has had numerous papers published which extend the algorithm to account for power bias at lower periods \citep{liu2007rectification,veleda2012cross} as well as being able to discern if the signal is standing or propagating \citep{2008SoPh..248..395S}.
    The Empirical Mode Decomposition has been extended, adding a step to the method where it does an ensemble averages \citep{wu2009ensemble}, better methods to deal with the edge effects with the spline fit \citep{zeng2004simple} as well as improved stopping criterion \citep{huang2008review}.
    It is important that the methods used to analyse signals and extract phase relations and amplitudes be made robuster in order to be certain in the period, phase and amplitudes of any oscillations found.
    If magneto-seismology requires these quantities to calculate the background properties of flux tubes, it is vital that any signal analysis method used is as robust as possible. 
    
    Finally, the observed UPWs within the magnetic pore is an interesting avenue of wave research.
    Here, there are many questions that need to be answered.
    To start, this observation indicated that the UPW events were fast sausage modes needs to be squared with the current literature.
    Either the interpretation is correct and UPWs can be either fast or slow waves or the analysis is incorrect.
    Further ground-based observations of magnetic pores would need to be conducted in order to understand UPWs within these structures.
    The author is only aware that RPWs have been only observed in sunspots , is the difference in wave mode due to a difference between sunspots and magnetic pores? 
    Does the presence of the penumbra cause this or would it be that the plasma-$\beta$ varies in a pore differently such that mode conversion leads to a fast magnetic-acoustic wave instead of a slow magnetic-acoustic wave?
    In Chapter 5, the region where the waves were observed was a small region of quiet chromosphere.
    Does the surrounding chromospheric atmosphere dictate if UPWs can be observed?
    Generally, symmetrical sunspots display concentric and clear RPWs, is that due to the magnetic field topology of the sunspot being much stronger than a magnetic pore, thus the observation of UPWs allows us to infer the magnetic field topology around sunspots and magnetic pores?
    Finally, do UPWs have a common source with the LOS oscillations and cross-sectional area oscillations observed in these magnetic structures, or are they the same wave that can be observed in several different ways?
    Solar physics has a long way to go and so does the research presented within this Thesis.